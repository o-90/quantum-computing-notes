\documentclass{article}

\usepackage[preprint]{./template/neurips_2020}

\usepackage[utf8]{inputenc} % allow utf-8 input
\usepackage[T1]{fontenc}    % use 8-bit T1 fonts
\usepackage{hyperref}       % hyperlinks
\usepackage{url}            % simple URL typesetting
\usepackage{booktabs}       % professional-quality tables
\usepackage{amsfonts}       % blackboard math symbols
\usepackage{nicefrac}       % compact symbols for 1/2, etc.
\usepackage{microtype}      % microtypography
\usepackage{amssymb}
\usepackage{physics}
\usepackage{amsmath}
\usepackage{tikz}
\usetikzlibrary{quantikz}

\title{Chapter 4: Quantum Circuits}

\author{
  John Martinez \\
  \texttt{john.r.martinez14@gmail.com} \\
}

\begin{document}
\maketitle

\section{Quantum Circuits}

%%% Section 4.1
\subsection{Quantum Algorithms}
What is a quantum computer good for? The promise of quantum computers is to
enable new algorithms which render feasible problems requiring exorbiant
resources for their solution on a classical computer.

%%% Section 4.2
\subsection{Single Qubit Operations}
Useful gates not mentioned yet:

  \begin{center}
    $
      T = \begin{bmatrix}
        1 & 0 \\
        0 & e^{i\pi / 4}
      \end{bmatrix};\mspace{10mu}
      S = \begin{bmatrix}
        1 & 0 \\
        0 & i
      \end{bmatrix}
    $
  \end{center}

A few useful algebraic facts to keep in mind are that
$H = \frac{(X + Z)}{\sqrt{2}}$ and $S = T^{2}$

The Pauli matrices give rise to three useful classes of unitary matrices when
they are exponentiated, the \emph{rotation operators} about the $\hat{x}$,
$\hat{y}$, and $\hat{z}$ axes, defined by the equations:

  \begin{center}
    $R_{x}(\theta) \equiv e^{-i\theta X / 2} = 
    \begin{bmatrix}
      \cos{\frac{\theta}{2}} & -i\sin{\frac{\theta}{2}} \\
      -i\sin{\frac{\theta}{2}} & \cos{\frac{\theta}{2}}
    \end{bmatrix}
    $
  \end{center}

  \begin{center}
    $R_{y}(\theta) \equiv e^{-i\theta Y / 2} = 
    \begin{bmatrix}
      \cos{\frac{\theta}{2}} & -\sin{\frac{\theta}{2}} \\
      \sin{\frac{\theta}{2}} & \cos{\frac{\theta}{2}}
    \end{bmatrix}
    $
  \end{center}

  \begin{center}
    $R_{z}(\theta) \equiv e^{-i\theta Z / 2} = 
    \begin{bmatrix}
      e^{-i\theta / 2} & 0 \\
      0 & e^{i\theta / 2}
    \end{bmatrix}
    $
  \end{center}


\emph{Theorem 4.1}:

Suppose $U$ is a unitary operation on a single qubit.
Then $\exists \alpha, \beta, \gamma, \delta \in \mathbb{R}$
such that
  \begin{center}
    $U = e^{i\alpha}R_{z}(\beta)R_{y}(\gamma)R_{z}(\delta)$.
  \end{center}

\emph{Proof}

Since $U$ is unitary, the rows and columns of $U$ are orthonomal, from which it
follow that $\exists \alpha, \beta, \gamma, \delta \in \mathbb{R}$ such that
\begin{center}
  $U = \begin{bmatrix}
  e^{i(\alpha-\beta/2-\delta/2)}\cos{(\gamma/2)} &
  -e^{i(\alpha-\beta/2+\delta/2)}\sin{(\gamma/2)} \\
  e^{i(\alpha+\beta/2-\delta/2)}\sin{(\gamma/2)} &
  e^{i(\alpha+\beta/2+\delta/2)}\cos{(\gamma/2)}
  \end{bmatrix}$
\end{center}

The above equation follows immediately from the definition of the rotational
matrices.

$\blacksquare$

%%% Section 4.3
\subsection{Controlled Operations}
"If $A$ is true, then do $B$". This is an example of a
\emph{controlled operation}.

A prototypical controlled operation is the controlled-$NOT$. The action of
the $CNOT$ is given by $\ket{c}\ket{t} \rightarrow \ket{c}\ket{t \oplus c}$.
The matrix representation of $CNOT$ is

\begin{center}
  $\begin{bmatrix}
    1 & 0 & 0 & 0 \\
    0 & 1 & 0 & 0 \\
    0 & 0 & 0 & 1 \\
    0 & 0 & 1 & 0
  \end{bmatrix}$
\end{center}

\begin{center}
  \begin{quantikz}
    \ket{c} & \qw & \ctrl{1} & \qw \\
    \ket{t} & \qw & \targ{} & \qw 
  \end{quantikz}
\end{center}

Suppose $U$ is an arbitrary single qubit unitary operation. A
\emph{controlled-U} operation is a two qubit operation. That is,
$\ket{c}\ket{t} \rightarrow \ket{c}U^{c}\ket{t}$

\begin{center}
  \begin{quantikz}
    \ket{c} & \qw & \ctrl{1} & \qw \\
    \ket{t} & \qw & \gate{U} & \qw 
  \end{quantikz}
\end{center}

Our immediate goal is to understand how to implement the controlled-\emph{U}
operation for arbitrary single qubit $U$, using only single qubit operations
and the \emph{CNOT} gate. Our strategy is a two-part procedure based upon the
decomposition $U = e^{i\alpha}AXBXC$ given in Corollary 4.2 on page 176.

\emph{Corollary 4.2}: Suppose $U$ is a unitary gate on a single qubit. The there
exists unitary operators, $A$, $B$, $C$ on a single qubit such that $ABC = I$
and $U = e^{i\alpha}AXBXC$, where $\alpha$ is some overall phase factor.

\begin{center}
\begin{quantikz}[align equals at=1.5,column sep=0.3cm]
  \qw & \ctrl{1} & \ghost{\begin{bmatrix}
      1 & 0 \\
      0 & e^{i\alpha}
    \end{bmatrix}} \qw \\
  \qw & \gate{U} & \qw 
\end{quantikz}{=}\begin{quantikz}[align equals at=1.5,column sep=0.3cm]
  \qw & \qw & \ctrl{1} & \qw & \ctrl{1} & \gate{
    \begin{bmatrix}
      1 & 0 \\
      0 & e^{i\alpha}
    \end{bmatrix}
  } & \qw \\
  \qw & \gate{C} & \targ{} & \gate{B} & \targ{} & \gate{A} & \qw
\end{quantikz}
\end{center}

Toffoli Gate
\begin{center}
\begin{quantikz}[align equals at=2.0]
  \qw & \ctrl{1} & \ghost{H} \qw \\
  \qw & \ctrl{1} & \ghost{H^{\dagger}} \qw \\
  \qw & \targ{}  & \ghost{H^{\dagger}} \qw
\end{quantikz}{=}
\begin{quantikz}[align equals at=2.0,column sep=0.2cm]  %,row sep=0.2cm]
  \qw & \qw & \qw & \qw & \ctrl{2} & \qw & \qw & \qw & \ctrl{2} & \qw
      & \ctrl{1} & \qw & \ctrl{1} & \gate{T} \\
  \qw & \qw & \ctrl{1} & \qw & \qw & \qw & \ctrl{1} & \qw & \qw
      & \gate{T^{\dagger}} & \targ{} & \gate{T^{\dagger}} & \targ{}
      & \gate{S} \\
  \qw & \gate{H} & \targ{} & \gate{T^{\dagger}} & \targ{} & \gate{T} & \targ{}
      & \gate{T^{\dagger}} & \targ{} & \gate{T} & \gate{H} & \qw & \qw
      & \qw
\end{quantikz}
\end{center}

Network implementation of the $C^{n}(U)$ operation for the case $n = 5$.

\begin{center}
\begin{quantikz}
\lstick[wires=5]{$control$}
  \ket{c_1} & \ctrl{1} & \qw & \qw & \qw & \qw & \qw & \qw & \qw & \ctrl{1} & \qw \\
  \ket{c_2} & \ctrl{4} & \qw & \qw & \qw & \qw & \qw & \qw & \qw & \ctrl{4} & \qw \\
  \ket{c_3} & \qw & \ctrl{3} & \qw & \qw & \qw & \qw & \qw & \ctrl{3} & \qw & \qw \\
  \ket{c_4} & \qw & \qw & \ctrl{3} & \qw & \qw & \qw & \ctrl{3} & \qw & \qw & \qw \\
  \ket{c_5} & \qw & \qw & \qw & \ctrl{3} & \qw & \ctrl{3} & \qw & \qw & \qw & \qw \\
\lstick[wires=4]{$work$}
  \ket{0} & \targ{} & \ctrl{1} & \qw & \qw & \qw & \qw & \qw & \ctrl{1} & \targ{} & \qw \\
  \ket{0} & \qw & \targ{} & \ctrl{1} & \qw & \qw & \qw & \ctrl{1} & \targ{} & \qw & \qw \\
  \ket{0} & \qw & \qw & \targ{} & \ctrl{1} & \qw & \ctrl{1} & \targ{} & \qw & \qw & \qw \\
  \ket{0} & \qw & \qw & \qw & \targ{} & \ctrl{1} & \targ{} & \qw & \qw & \qw & \qw \\
\lstick[wires=1]{$target$}
  \qw & \qw & \qw & \qw & \qw & \gate{U} & \qw & \qw & \qw & \qw & \qw
\end{quantikz}
\end{center}

%%% Section 4.4
\subsection{Measurement}
There are two important principles that it is worth bearing in mind about
quantum circuits.

\textbf{Principle of deferred measurement}: Measurements can always be moved
from an intermediate stage of a quantum circuit to the end of the circuit;
if the measurement results are used at any stage of the circuit, then the
classically controlled operations can be replaced by conditional quantum
operations.

\emph{Quantum Teleportation circuit}
\begin{center}
\begin{quantikz}
  \ket{\psi} & \ctrl{1} & \gate{H} & \qw & \qw & \ctrl{2} & \meter{} \\
\lstick[wires=2]{$\ket{\beta_{00}}$}
  \qw & \targ{} & \qw & \qw & \ctrl{1} & \qw & \meter{} \\
  \qw & \qw & \qw & \qw & \gate{X} & \gate{Z} & \qw\rstick[wires=1]{$\ket{\psi}$}
\end{quantikz}
\end{center}


\textbf{Principle of implicit measurement}: Without loss of generality,
any unterminated quantum wires (quibits which are not measured) at the
end of a quantum circuit may be assumed to be measured.

%%% Section 4.5
\subsection{Universal Quantum Gates}
\subsubsection{Two-Level Unitary Gates are Universal}
Consider a Unitary matrix $U$ which acts on a $d$-dimensional Hilbert space.
$U$ may be decomposed into a product of \emph{two-level unitary matrices}.
Suppose $U$ has the form

\begin{center}
  $
    U = \begin{bmatrix}
      a & d & g \\
      b & e & h \\
      c & f & j
    \end{bmatrix}
  $
\end{center}

We will find two-level unitary matrices $U_1,\dots,U_3$ such that
\begin{center}$U_3 U_2 U_1 U = I.$\end{center}

It follows that
\begin{center}$U = U_{1}^{\dagger}U_{2}^{\dagger}U_{3}^{\dagger}.$\end{center}

$U_1$, $U_2$, and $U_3$ are all two-level unitary matrices, and it is easy to
see that their inverses, $U_{1}^{\dagger}$, $U_{2}^{\dagger}$, and
$U_{3}^{\dagger}$ are also two-level unitary matrices.

\subsubsection{Single Qubit and $CNOT$ Gates are Universal}
\end{document}
