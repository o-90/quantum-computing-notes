\documentclass{article}

\usepackage[preprint]{./template/neurips_2020}

\usepackage[utf8]{inputenc} % allow utf-8 input
\usepackage[T1]{fontenc}    % use 8-bit T1 fonts
\usepackage{hyperref}       % hyperlinks
\usepackage{url}            % simple URL typesetting
\usepackage{booktabs}       % professional-quality tables
\usepackage{amsfonts}       % blackboard math symbols
\usepackage{nicefrac}       % compact symbols for 1/2, etc.
\usepackage{microtype}      % microtypography
\usepackage{amssymb}
\usepackage{physics}
\usepackage{amsmath}
\usepackage{amssymb}
\usepackage{esvect}
\usepackage{tikz}
\usetikzlibrary{quantikz}

\title{Chapter 4: Quantum Circuits}

\author{
  John Martinez \\
  \texttt{john.r.martinez14@gmail.com} \\
}

\begin{document}
\maketitle

% \subsection{}
% \textbf{Solution}
% $\blacksquare$

\section{Quantum Circuits}

% 4.1 ------------------------------------------------------------------------
\subsection{Exercise}
% \textbf{Solution}
% $\blacksquare$

% 4.2 ------------------------------------------------------------------------
\subsection{Exercise}
Let $x \in \mathbb{R}$ and $A$ a matrix such that $A^{2} = I$. Show that
\begin{center}
  $e^{iAx} = I\cos{x} + i A\sin{x}$
\end{center}

\textbf{Solution}

From the definition of the \emph{rotational operator} $\hat{y}$,

\begin{center}
  $R_a(-2\theta) = \cos{\theta}I + i\sin{\theta}A = e^{iA\theta}$
\end{center}
$\blacksquare$

% 4.3 ------------------------------------------------------------------------
\subsection{Exercise}
Show that, up to global phase, the $\frac{\pi}{8}$
gate satisfies $T = R_{z}(\frac{\pi}{4})$

\textbf{Solution}

$
  R_{z}(\frac{\pi}{4}) = e^{-i \pi Z / 8} = 
  \begin{bmatrix}
    e^{-i \pi / 8} & 0 \\
    0 & e^{i \pi / 8}
  \end{bmatrix} = 
  e^{-i \pi / 8}\begin{bmatrix}
    1 & 0 \\
    0 & e^{i \pi / 4}
  \end{bmatrix} = T$

$\blacksquare$

% 4.4 ------------------------------------------------------------------------
\subsection{Exercise}
Express Hadamard gate $H$ as a product of $R_{x}$ and $R_{z}$ rotations and
$e^{i\phi}$ for some $\phi$

\textbf{Solution}

Let $c = \frac{1}{\sqrt{2}}e^{i\phi/2}$.

Then
  $cR_{x}(\phi)R_{z}(\phi) =
  c(e^{-i\phi Y / 2} e^{-i\phi Z / 2}) = 
  c(e^{-i\frac{\phi}{2}(X + Z)}) =
  c(e^{-i\frac{\phi}{\sqrt{2}} H}) =
  \frac{1}{\sqrt{2}}\begin{bmatrix}
    1 & 1 \\
    1 & e^{i\phi}
  \end{bmatrix}$

$\therefore \mspace{20mu} \phi = (2k+1)\pi; \mspace{20mu} \forall k \in{\mathbb{N}}$

$\implies \frac{1}{\sqrt{2}}e^{-(2k+1)\pi/2}R_{x}((2k+1)\pi)R_{z}((2k+1)\pi) =
  \frac{1}{\sqrt{2}}\begin{bmatrix}
    1 & 1 \\
    1 & e^{i(2k+1)\pi}
  \end{bmatrix} = 
  \frac{1}{\sqrt{2}}\begin{bmatrix}
    1 & 1 \\
    1 & -1
  \end{bmatrix} = H$.

$\blacksquare$

% 4.5 ------------------------------------------------------------------------
\subsection{Exercise}
Prove that $(\hat{n}, \vv{\sigma})^{2} = I$, and use this to verify that
\begin{center}
  $R_{\hat{n}}(\theta) \equiv e^{-i\theta\hat{n}\vv{\sigma}/2} =
  \cos{\frac{\theta}{2}}I - i\sin{\frac{\theta}{2}}(n_{x}X + n_{y}Y + n_{z}Z)$
\end{center}

where $\vv{\sigma}$ denotes the three component vector $(X, Y, Z)$ of Pauli
matrices and where $\hat{n}$ is a real unit vector.

\textbf{Solution}

$\hat{n} = (n_x, n_y, n_z) \in \mathbb{R}^{3}$

$\vv{\sigma} = (X, Y, Z)$

$
  A^{2} = (\hat{n}\vv{\sigma})^{2} = (n_{x}X + n_{y}Y + n_{z}Z)^{2} = 
  \begin{bmatrix}
    n_z & n_x - in_y \\
    n_x + in_y & -n_z
  \end{bmatrix}^{2}
$

$
\mspace{20mu} = \begin{bmatrix}
    n_{x}^{2} + n_{y}^{2} + n_{z}^{2} & 0 \\
    0 & n_{x}^{2} + n_{y}^{2} + n_{z}^{2}
  \end{bmatrix}
$

We are given $\hat{n}$ is a unit vector $\implies n_{x}^{2} + n_{y}^{2} + n_{z}^{2} = 1$

$\therefore \mspace{20mu} A^{2} = I$

$\blacksquare$

From \textbf{Exercise 1.4}, we know $e^{iAx} = \cos{(x)}I + i\sin{(x)A}$ and
using $A^{2} = I$

$A = \hat{n}\vv{\sigma}$ and $x = -\theta / 2$ so

$R_{\hat{n}}(\theta) = \cos{(\theta/2)} - i\sin{(\theta/2)}\hat{n}\vv{\sigma}$

$\implies \cos{(\frac{\theta}{2})} - i\sin{(\frac{\theta}{2})}(n_xX + n_yY + n_zZ)$


$\blacksquare$
% 4.6 ------------------------------------------------------------------------
\subsection{Exercise-Block Sphere Interpretation of Rotations}
$R_{\hat{n}}(\theta)$ operators are referred to as rotation operators is the
following fact, which you are to prove. Suppose a single qubit has a state
represented by the Block vector $\vv{\lambda}$. Then the effect
of the rotation $R_{\hat{n}}(\theta)$ on the state is to rotate it by an
angle $\theta$ about the $\hat{n}$ axis of the Block sphere. This fact
explains the rather mysterious looking factor of two in the definition
of the rotation matrices.

% \textbf{Solution}
% $\blacksquare$

% 4.7 ------------------------------------------------------------------------
\subsection{Exercise}
Show that $XYX = -Y$ and use this to prove that $XR_{y}(\theta)X = R_{y}(-\theta)$

\textbf{Solution}

$XYX =
  \begin{bmatrix}0 & 1 \\ 1 & 0\end{bmatrix}
  \begin{bmatrix}0 & -i \\ i & 0\end{bmatrix}
  \begin{bmatrix}0 & 1 \\ 1 & 0\end{bmatrix} = 
  \begin{bmatrix}i & 0 \\ 0 & -i \end{bmatrix}
  \begin{bmatrix}0 & 1 \\ 1 & 0\end{bmatrix} = 
  \begin{bmatrix}0 & i \\ -i & 0\end{bmatrix} = -Y
$

$
X R_{y}(\theta)X = 
\begin{bmatrix}0 & 1 \\ 1 & 0\end{bmatrix}
\begin{bmatrix}
  \cos{(\frac{\theta}{2})} & -\sin{(\frac{\theta}{2})} \\
  \sin{(\frac{\theta}{2})} & \cos{(\frac{\theta}{2})}
\end{bmatrix}
\begin{bmatrix}0 & 1 \\ 1 & 0\end{bmatrix} =
\begin{bmatrix}
  -\sin{(\frac{\theta}{2})} & \cos{(\frac{\theta}{2})} \\
  \cos{(\frac{\theta}{2})} & \sin{(\frac{\theta}{2})}
\end{bmatrix}
\begin{bmatrix}0 & 1 \\ 1 & 0\end{bmatrix} =
\begin{bmatrix}
  \cos{(\frac{\theta}{2})} & \sin{(\frac{\theta}{2})} \\
  -\sin{(\frac{\theta}{2})} & \cos{(\frac{\theta}{2})}
\end{bmatrix} = R_{y}(-\theta)
$

$\blacksquare$

% 4.8 ------------------------------------------------------------------------
\subsection{Exercise}
An arbitrary single qubit unitary operator can be written in the form
\begin{center}
  $U = e^{\alpha i}R_{\hat{n}}(\theta)$
\end{center}
for some $\alpha, \theta \in \mathbb{R}$ and a three-dimensional unit vector $\hat{n}$.

1. Prove this fact

2. Find values for $\alpha$, $\theta$, and $\hat{n}$ giving the Hadamard gate $H$.

3. Find values for $\alpha$, $\theta$, and $\hat{n}$ giving the phase gate

\begin{center}
  $S = 
    \begin{bmatrix}
      1 & 0 \\
      0 & i
    \end{bmatrix}$.
\end{center}

\textbf{Solution}

1. A operator will be unitary if $U^{\dagger}U = I$ holds.

\begin{center}
  $U = \cos{(\theta/2)}I - i\sin{(\theta/2)}A$
\end{center}

and

\begin{center}
  $U^{\dagger} = \cos{(\theta/2)}I + i\sin{(\theta/2)}A^{\dagger}$
\end{center}

where

\begin{center}
  $A = (\hat{n},\vv{\sigma}) = (n_{x}X+n_{y}Y+n_{z}Z)$
\end{center}

Then

\begin{center}
  $U^{\dagger}U = \cos^{2}{(\theta/2)}I + \sin^{2}{(\theta/2)}A^{\dagger}A$
\end{center}

From \textbf{Exercise 5}, we know $A = A^{\dagger} \implies A^{\dagger}A = I$ 

$\therefore$

\begin{center}
  $U^{\dagger}U = I(\cos^{2}{(\theta/2)} + \sin^{2}{(\theta/2)}) = I$
\end{center}

$\blacksquare$

%%% 4.9 ----------------------------------------------------------------------
\subsection{Exercise}

%%% 4.10 ---------------------------------------------------------------------
\subsection{Exercise}

%%% 4.11 ---------------------------------------------------------------------
\subsection{Exercise}

%%% 4.12 ---------------------------------------------------------------------
\subsection{Exercise}

%%% 4.13 ---------------------------------------------------------------------
\subsection{Exercise}

%%% 4.14 ---------------------------------------------------------------------
\subsection{Exercise}

%%% 4.15 ---------------------------------------------------------------------
\subsection{Exercise}

%%% 4.16 ---------------------------------------------------------------------
\subsection{Exercise}

%%% 4.17 ---------------------------------------------------------------------
\subsection{Exercise}

Construct a $CNOT$ gate from one controlled-$Z$ gate and two Hadamard
gates, where

\begin{center}
  $Z = \begin{bmatrix}
    1 & 0 & 0 & 0 \\
    0 & 1 & 0 & 0 \\
    0 & 0 & 1 & 0 \\
    0 & 0 & 0 & -1
  \end{bmatrix}$
\end{center}

\textbf{Solution}

\begin{center}
\begin{quantikz}
\qw & \qw & \ctrl{1} & \qw & \qw \\
\qw & \gate{H} & \gate{Z} & \gate{H} & \qw
\end{quantikz}{=}
\begin{quantikz}
\qw & \ctrl{1} & \qw \\
\qw & \targ{} & \qw
\end{quantikz}
\end{center}

$\blacksquare$

%%% 4.18 ---------------------------------------------------------------------
\subsection{Exercise}

Show that

\begin{center}
\begin{quantikz}
\qw & \ctrl{1} & \qw \\
\qw & \gate{Z} & \qw
\end{quantikz}{=}
\begin{quantikz}
\qw & \gate{Z} & \qw \\
\qw & \ctrl{-1} & \qw
\end{quantikz}
\end{center}

\textbf{Solution}

$\blacksquare$

\end{document}
