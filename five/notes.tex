\documentclass{article}

\usepackage[preprint]{./template/neurips_2020}

\usepackage[utf8]{inputenc} % allow utf-8 input
\usepackage[T1]{fontenc}    % use 8-bit T1 fonts
\usepackage{hyperref}       % hyperlinks
\usepackage{url}            % simple URL typesetting
\usepackage{booktabs}       % professional-quality tables
\usepackage{amsfonts}       % blackboard math symbols
\usepackage{nicefrac}       % compact symbols for 1/2, etc.
\usepackage{microtype}      % microtypography
\usepackage{amssymb}
\usepackage{physics}
\usepackage{amsmath}
\usepackage{tikz}
\usetikzlibrary{quantikz}

\title{Chapter 5: The Quantum Fourier Transform and its Applications}

\author{
  John Martinez \\
  \texttt{john.r.martinez14@gmail.com} \\
}

\begin{document}
\maketitle

\section{The Quantum Fourier Transform and its Applications}

%%% Section 5.1
\subsection{The Quantum Fourier Transform}
The \emph{discrete fourier transform} takes as input a vector of complex numbers
$x_0,\dots,X_{N-1}$ where the length $N$ of the vector is a fixed parameter. It
outputs the transformed data, a vector of complex numbers $y_{0},\dots,y_{N-1}$,
defined by
\begin{center}
  $y_{N} \equiv \frac{1}{\sqrt{N}}\displaystyle\sum_{j=0}^{N-1} x_{j}e^{2\pi i j k/N}$.
\end{center}

The \emph{quantum fourier transform} is exactly the same transformation,
although the conventional notation for the quantum fourier transform is
somewhat different. The quantum fourier transform on an orthonormal basis
$\ket{0},\dots,\ket{N-1}$ is defined to be linear operator with the followin
action on the basis states:
\begin{center}
  $\ket{j} \rightarrow
  \frac{1}{\sqrt{N}}\displaystyle\sum_{k=0}^{N-1}e^{2\pi i j k /N}\ket{k}$.
\end{center}

The action on an arbitrary state maybe written
\begin{center}
  $\displaystyle\sum_{j=0}^{N-1}x_j\ket{j} \rightarrow
  \displaystyle\sum_{k=0}^{N-1}y_k\ket{k}$.
\end{center}

It is not ovbious, but this transformation is a unitary transformation.
\end{document}
